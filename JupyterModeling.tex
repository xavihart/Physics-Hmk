\documentclass{article}
\usepackage[utf8]{inputenc}
\usepackage[UTF8]{ctex}
\usepackage{amsthm}
\usepackage{amsmath}
\usepackage{amssymb}
\usepackage{graphicx}
\usepackage{hyperref}
\usepackage[table]{xcolor}
\usepackage{fancyhdr}
\usepackage{lastpage}
\usepackage{pythonhighlight}
\usepackage{subfigure}
\usepackage{fancyhdr}
\usepackage[superscript]{cite}
\usepackage{geometry}

\geometry{left=4cm,right=4cm,top=1.5cm,bottom=1.5cm}
\begin{document}
\title{木星磁场的建模与计算}
\author{薛昊天 518021910506}
\date{}

\maketitle
\begin{abstract}
   木星作为太阳系最大的行星,拥有最强的行星磁场。木星磁场是木星探测的基本环境之一,因此对木星磁场的建模很有意义。本文通过借鉴国际地磁场的建模计算过程,在木星上类比计算出磁场强度。\\
  \textbf{关键词:木星,磁场,建模}
  
\end{abstract}




\section{建模过程}
 \subsection{基本假设}
   在本文中,我们假设木星是一个规则球体,半径为$R_J$,参考系使用 Right-handed System III
   (S3RH), 如图1:在这一参考系下,木星表面一个点的经度$\lambda_{RH}$随着自传而减小。
   同时我们假设木星内部的磁场是由恒定电场产生的稳恒磁场,是一个有势无旋场。
   
\begin{figure}
    \centering
    \includegraphics[scale=1.2]{./figure/S3RH.PNG}
    \caption{S3RH木星坐标系}
    \label{fig:my_label}
\end{figure}   

   \subsection{公式}
   首先在麦克斯韦方程组中,在稳恒场中有:
   
   \begin{equation} 
       \nabla \times E = \frac{\partial{B}}{\partial{t}} 
   \end{equation}

   而在假设的稳恒电厂下,有 $E=-\nabla{\varphi}$。在S3RH参考系下,有$\varphi=U(r,\theta,\lambda)$,代入式(1)有:
   
   \begin{equation} 
       \nabla^{2}\varphi=\nabla^{2}U(r,\theta,\lambda)=0
   \end{equation}
   这是一个球坐标下的拉普拉斯方程,表达式可以求出是:

    \begin{equation}
        \frac{1}{r^{2}}\frac{\partial}{\partial{r}}(r^2\frac{\partial{U}}{\partial{r}})+\frac{1}{r^2sin\theta}\frac{\partial}{\partial\theta}(sin\theta\frac{\partial{U}}{\partial\theta}) + \frac{1}{r^2sin^2\theta}\frac{\partial^{2}U}{\partial{\lambda}^2}=0
   \end{equation}
   
分离变量r,有:
     \begin{equation} 
      U(r,\theta,\lambda) = R(r)Y(\theta, \lambda)
   \end{equation}
   
代入(3)变形可以得到:
\begin{equation} 
      \frac{1}{R}\frac{d}{dr}(r^2\frac{dR}{dr}) = \frac{1}{sin\theta{Y}}\frac{\partial}{\partial\theta}(sin\theta\frac{\partial{Y}}{\partial{\theta}})-\frac{1}{Y}\frac{1}{r^2sin\theta}\frac{\partial^2Y}{\partial\lambda^2}   
   \end{equation}
   
 上面等式的左边是关于r的函数,右边是关于$\theta,\lambda$的函数,所以由自变量的任意性可以知道等式两边都恒等与某一个常数,记为$l(l+1)$,则可以将上述方程拆分成两个方程:
 
 \begin{equation}
         \frac{d}{dr}(r^2\frac{dR}{dr})-l(l+1)R=0      
 \end{equation}
 
 \begin{equation}
        \frac{1}{sin\theta}\frac{\partial}{\partianl\theta}(sin\theta\frac{\partial{Y}}{\partial\theta})+\frac{1}{sin^2\theta}\frac{\partial^2Y}{\partial\lambda^2}+l(l+1)Y=0
 \end{equation}
 
   其中方程(6)是一个欧拉型常微分方程,它的解是:
   
  \begin{equation}
      R(r)=Cr^{l} + Dr^{-l-1}
  \end{equation}
  
  方程(7)仍然是球函数方程,可以进一步分离变量:$Y(\theta,\lambda)=\Gamma(\theta)\Phi(\lambda)$

 代入(7)中,将$\theta$和$\lambda$有关的式子分别移到等式两侧得到:
 
 \begin{equation}
     \frac{sin\theta}{\Gamma}\frac{d}{d\theta}(sin\theta\frac{d\Gamma}{d\theta})+l(l+1)sin^2\theta=-\frac{1}{\Phi}\frac{d^2\Phi}{d^2\lambda}
 \end{equation}
\section{计算过程}
同样可以知道左右一定恒等与某一个常数



\section{总结}

\end{document}
