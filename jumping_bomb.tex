\documentclass[UTF8]{ctexart}

\usepackage{listings}

\CTEXsetup[format={\Large\bfseries}]{section}


\usepackage{amsthm}
\usepackage{amsmath}
\usepackage{amssymb}
\usepackage{graphicx}
\usepackage{hyperref}
\usepackage[table]{xcolor}
\usepackage{fancyhdr}
\usepackage{lastpage}
\usepackage{pythonhighlight}
\usepackage{subfigure}
\usepackage{fancyhdr}
\usepackage[superscript]{cite}
\usepackage{geometry}

\geometry{left=4cm,right=4cm,top=1.5cm,bottom=1.5cm}


\begin{document}
\title{\textbf{跳跃式炸弹的建模与分析}}
\author{薛昊天  518021910506}
\date{2019年6月15日}
\maketitle
\begin{abstract}
  二战时期,英国皇家空军为了打击德军鲁尔区的水坝$^{[1]}$,使用了跳跃式炸弹。这种炸弹在投放后会在水面上跳跃前进,以不被德军的水下鱼雷所拦截,
  最后可以水坝底部爆炸,通过冲击波成功摧毁水坝。本文主要对跳跃式炸弹进行建模,分析其合理性并结合具体数据进行模拟计算。

\end{abstract} 

 \section{概述}
      “跳跃式炸弹”的优势在于:在精确计算好投弹的距离和高度后,飞机低飞投射出的炸弹在水面上
      跳跃旋转前进,于是可以成功地躲避德军的水下鱼雷的拦截$^{[2]}$,最后在到达水坝时滚入水坝底部,通过水下爆炸形成的冲击波摧毁水坝。下面我
      我将对炸弹在水上的运动进行建模,分析其合理性,并计算投弹的距离。
  

 \section{建模原理以及合理性}
 \subsection{建模基本原理}

  建模涉及到以下几个流体力学公式:

       \begin{equation}
           p+\frac{1}{2}\rho{v^2}=const
       \end{equation}
          (1)式是伯努利方程$^{[3]}$,p是流体某处压力,$\rho$是液体密度,v是
       液体在这一点的流速。
       \begin{equation}
          f=\mu{S}\frac{dv_r}{dr} 
       \end{equation}
       (2)式是液体粘滞阻力方程$^{[3]}$,其本质是物体在流体中的运动带动周围
       水的运动,造成水层之间的摩擦,形成阻力,其中S是接触面积,$\mu$
        是流体的动力粘度,$\frac{dv_r}{dr}$表示在速度的
        垂直于表面方向上单位长度上的速度增量。
  
   
     
   \subsection{建模前提}
     在本文计算中,做如下两个假设:\textcircled{1}用伯努利方程计算水对物体的压力时,假设物体周围的水膜厚度不计。\textcircled{2}在物体在水中运动时,带动周围的一层水膜运动,假设这层水膜的厚度是
       固定的$\Delta{y}$,且水膜中水的流速随着原理物体均匀地从$v_r$下降到0,即假设
       水是不可压缩的粘性液体。

 \section{轨迹建模过程}
   \subsection{分析某一次跳跃}
    假设某一时刻(见图一),炸弹部分浸入水中,这时炸弹受重力G,浮力$F_0$,水对炸弹
    的粘滞阻力f,流速导致的压强差造成的向上的升力$F_l$。\\
    由公式(1)可知圆柱体入水部分表面每一点的压强为:
    \begin{equation}
p=\frac{1}{2}\rho{v^2}
    \end{equation}
     
    \indent 又由公式(2)以及假设1可知,水中部分每一点受水的压强为$\frac{1}{2}\rho{v^2}$,其中v等于
    $v={v_x}+\omega{R}$,对进入水中部分积分可得(结果发现只有竖直分量) :

  \begin{equation}
      F_l=\int_{s}\vec{P}ds=\rho{v^2}RLsin\theta
  \end{equation}

  再由公式(2)以及假设2,对每一点的粘滞阻力积分得:

     
  
  \begin{equation}
    f=2\mu{S}\frac{v}{\Delta{y}}sin\theta
\end{equation}
  其中接触面积$S=2\theta{RL}$,L为炸弹长度,m为
  炸弹质量。
浮力容易计算得到:
  \begin{equation}
    F_0=L(\theta-\frac{1}{2}sin2\theta)R^2\rho{g}
\end{equation}
  
  \begin{figure}[t]
    \centering 
    \includegraphics[width=0.8\textwidth]{figure1.PNG}    
    \caption{炸弹水面受力示意图}
  \end{figure}




   \indent 所以综合以上列出物体的运动学方程(取向右,向上,逆时针为正):

   \begin{equation}
    \begin{cases}
      m\ddot{x}=-2\mu{S}\frac{v}{\Delta{y}}sin\theta \\
      m\ddot{y}=-G+\rho{v^2}RLsin\theta+L(\theta-\frac{1}{2}sin2\theta)R^2\rho{g}\\
      M=fR=-\frac{1}{2}mR^2\dot{\omega}\\
      \Delta{y}=-sin\theta{R\Delta{\theta}}\\
      v=\dot{x}+\omega{R}
    \end{cases}
   \end{equation}

由能量守恒可知,设第i次的入水和出水的x,y方向速度为${v_x}^{(i,in)},{v_y}^{(i,in)},{v_x}^{(i,out)},{v_y}^{(i,out)}$,有:
\begin{equation}
  \begin{cases}
    {v_x}^{(i+1,in)}={v_x}^{(i,out)}\\
    {v_y}^{(i+1,in)}=-{v_y}^{(i,out)}
  \end{cases}
\end{equation}

  移动距离x=$\sum{\Delta{x_i}}=\sum{\int_{t_i}^{t_{i+1}}{{v_x(t)dt}}}$


   \subsection{最后的轨迹}
   由[1],炸弹的形状是长2.5米,直径1.5米的圆柱体,重量是3500公斤,飞机
   飞行速度335km/s,投射高度18米,炸弹转速大概是20rad/s。
   所以第一次的入水的${v_x}^{(1,in)}=93m/s,{v_y}^{(1,in)}=19.0m/s$。
    \indent 方程组(7)是二阶微分方程组且没有初等形式的解,但是从方程
    的形式我们可以运动状态大致为$^{[1]}$:
    \begin{figure}
      \centering
      \includegraphics[width=0.8\textwidth]{figure2.PNG}
      \caption{炸弹运动轨迹$^{[1]}$}
    \end{figure}

    \indent  再进行近似计算,假设$\theta$取平均值为$\frac{1}{2}\pi$那么先对
    x方向的微分方程简化可得:
    \begin{equation}
      \begin{cases}
           m\ddot{x}=-2\mu\frac{2\theta{RL}}{\Delta{y}}({\dot{x}+\omega{R}})\\
           -\frac{1}{2}mR^2\dot{\omega}=2R\pi{R^2}\mu{L}{\frac{\omega{R}+\dot{x}}{\Delta{y}}}
      \end{cases}
    \end{equation} 
两个式子相除再对时间积分可以得到:  $\dot{x}=\frac{1}{2}R\omega$,带回上一个方程组的第一个式子里,可以解除
再将x带回(7)的第二个式子中可以解析,结果为(x(0)=0,y(0)=R);
\begin{equation}
  \begin{cases}
      x=-\frac{1}{k}e^{-kt}+v_x(0)t\\
      y\approx(-g+\frac{L^2\rho{R^2}g\pi}{2m})t^2+9R\rho{L}v_x^2(0)t+R\\
      k={\frac{6\mu{\pi}RL}{m\Delta{y}}}
  \end{cases}
\end{equation}
代入数据可以求出每一次跳跃的时间为

 \section{结果分析}

  

\subsection{误差分析}




  
  \begin{tabular}{|c|c|c|}
   
\hline

计算步骤&产生误差的原因&误差程度\\
\hline 
(4)&水膜的厚度不可以忽略&小\\
\hline 
(5)&水膜之间的速度变化非线性&小\\
\hline
(7)&计算水中平动产生的粘滞阻力流体速度变化&中\\
\hline
(9)&计算轨迹解析式时$\theta$变化&较大 \\
\hline
(9)&计算y的解析计算式时省去相对小量&中\\
\hline 
  
 
  \end{tabular}


  \subsection{防御措施建议}
     考虑到炸弹是在近水面上跳跃前进的,所以距离水面较远的水下鱼雷
     拦截是失效的。针对这种跳跃式炸弹,一方面可以考虑在近水面增加拦截网,
     在炸弹为到达水坝之前就拦截炸弹,
     或者在水坝坝面水下一定深度增设拦截装置,以不让炸弹滚到堤坝底部爆炸,以减少损失。


    \section{致谢}
    感谢老师给予的启发性建议,感谢组员在论文排版和思路上给予的支持与帮助,让我
    顺利写完这篇论文,
    由于作者本人的学术水平有限,
    所写论文难免有不足之处,恳请老师和助教批评和指正!
      

 \begin{thebibliography}{1}

  \bibitem{Ben-Shimon2015RecSys}
  Ian A. Whalley1
Palm Springs.
Projected upkeep - A Review of the WWII dambuster weapon,2002
\bibitem{}
Michael Anderson,The dam buster,1955
\bibitem{}
 赵凯华,新概念物理教程(力学),2004

    
  
  \end{thebibliography}
  

\end{document}




